\documentclass{article}
\usepackage[utf8]{inputenc} 
\usepackage[spanish,es-tabla,es-nodecimaldot]{babel}
\usepackage{graphicx}
\usepackage{csvsimple}
\usepackage{authblk}
\usepackage{cprotect}
\usepackage{floatrow}
\floatsetup[table]{capposition=top}
\usepackage[caption=false]{subfig}
\usepackage{booktabs}
\usepackage{hyperref}
\usepackage{lscape}
\usepackage{gensymb}
\usepackage[a4paper,top=3cm,bottom=2cm,left=3cm,right=3cm,marginparwidth=1.75cm]{geometry}
\usepackage{siunitx}
\usepackage[square,sort,comma,numbers]{natbib}
\usepackage[tableposition=top]{caption}
\usepackage[nottoc,numbib]{tocbibind}
\usepackage{pythontex}

\author{Francisco Chávez y Nepo Rojas}

\title{Índice de Gini en México}

\begin{document}

\begin{pycode}
\end{pycode}

\maketitle
\begin{abstract}
\end{abstract}

\section*{Introducción}
Corrado Gini (Motta di Livenza, 23 May 1884 – Rome, 13 March 1965) was an Italian statistician,
demographer and sociologist who developed the Gini coefficient, a measure of the income inequality
in a society.

El coeficiente de Gini es una medida de la desigualdad ideada por el estadístico italiano Corrado
Gini. Normalmente se utiliza para medir la desigualdad en los ingresos, dentro de un país, pero
puede utilizarse para medir cualquier forma de distribución desigual. El coeficiente de Gini es un
número entre 0 y 1, en donde 0 se corresponde con la perfecta igualdad (todos tienen los mismos
ingresos) y donde el valor 1 se corresponde con la perfecta desigualdad (una persona tiene todos los
ingresos y los demás ninguno). El índice de Gini es el coeficiente de Gini expresado en referencia a
100 como máximo, en vez de 1, y es igual al coeficiente de Gini multiplicado por 100. Una variación
de dos centésimas del coeficiente de Gini (o dos unidades del índice) equivale a una distribución de
un 7\% de riqueza del sector más pobre de la población (por debajo de la mediana) al más rico (por
encima de la mediana).

Aunque el coeficiente de Gini se utiliza sobre todo para medir la desigualdad en los ingresos,
también puede utilizarse para medir la desigualdad en la riqueza. Este uso requiere que nadie
disponga de una riqueza neta negativa.\cite{carpenter2000}

\section*{Metodología}
\begin{table}[H]
\centering
\caption{Ingreso promedio nacional por decil (?`para qué año?}
\label{table:diferenciasTasas}
    \csvautobooktabular[respect all]{ingreso_decil.csv}
\end{table}

\begin{table}[H]
\centering
\caption{Ingreso promedio por entidad federativa (¿para cuál año?)}
\label{table:diferenciasTasas}
    \csvautobooktabular[respect all]{ingreso_estado.csv}
\end{table}

\begin{figure}[H]
    \includegraphics[scale=0.5]{boxplot_deciles.png}
    \caption{Diagrámas de cajas y bigotes para los ingresos nacionales para decil.}
    \label{fig:serieTasasVocalizaciones}
\end{figure}

\subsection*{Análisis}


\section*{Resultados}
\begin{table}[H]
\caption{Índice de Gini por entidad federativa.}
\centering
\label{table:diferenciasTasas}
    \csvautobooktabular[respect all]{gini_nacional.csv}
\end{table}

\bibliography{../references/tamano_poblacional.bib} 
\bibliographystyle{apalike}

\end{document}